\documentclass{beamer}
\usepackage{listings}
\usepackage{setspace}
\usepackage[utf8]{inputenc}
\usepackage{pgfplots}
\usepackage{lmodern}
\usepackage{newverbs}
\usepackage{natbib}
\pgfplotsset{compat=1.13}
\setlength\parindent{20pt}
\usetheme{Warsaw}
\newcommand\tab[1][1cm]{\hspace*{#1}}
\usepackage{array, longtable}
\usepackage{tabu}
\usepackage{xcolor}
\usepackage{colortbl}
\usepackage{amsmath}

\setbeamercolor{normal text}{fg=white,bg=black!90}
\setbeamercolor{structure}{fg=white}
\setbeamercolor{alerted text}{fg=red!85!black}
\setbeamercolor{item projected}{use=item,fg=black,bg=item.fg!35}
\setbeamercolor*{palette primary}{use=structure,fg=structure.fg}
\setbeamercolor*{palette secondary}{use=structure,fg=structure.fg!95!black}
\setbeamercolor*{palette tertiary}{use=structure,fg=structure.fg!90!black}
\setbeamercolor*{palette quaternary}{use=structure,fg=structure.fg!95!black,bg=green!80}
\setbeamercolor*{framesubtitle}{fg=white}
\setbeamercolor*{block title}{parent=structure,bg=black!60}
\setbeamercolor*{block body}{fg=gray,bg=black!10}
\setbeamercolor*{block title alerted}{parent=alerted text,bg=black!15}
\setbeamercolor*{block title example}{parent=example text,bg=black!15}
\title{Proyecto 4: Otro SIMPLEX m\'as}
\subtitle{Investigaci\'on de Operaciones \newline Semestre 1}
\author{Manrique J. Dur\'an V\'asquez - Randy M. Morales Gamboa}
\institute{Instituto Tecnol\'ogico de Costa Rica}
\date{\today}
\newverbcommand{\kwverb}{\color{red}}{}
\newverbcommand{\smverb}{\color{cyan}}{}
\newverbcommand{\idverb}{\color{brown}}{}
\newverbcommand{\assverb}{\color{orange}}{}
\newverbcommand{\cverb}{\color{pink}}{}
\newverbcommand{\opverb}{\color{green}}{}
\newverbcommand{\slverb}{\color{olive}}{}
\begin{document}
\begin{frame}
\maketitle
\end{frame}\begin{frame}
\frametitle{Algoritmo SIMPLEX}El algoritmo SIMPLEX fue inventado por George Dantzig en 1947. El SIMPLEX ayuda a encontrar la soluci\'on \'optima de un problema de programaci\'on lineal. Utiliza operaciones sobre matrices para encontrar la soluci\'on \'optima o determinar que el problema no tine soluci\'on.\newline Parte de un \textbf{v\'ertice} de la regi\'on factible y se mueve a \textbf{v\'ertices adyacentes} que mejoren lo encontrado hasta encontrar la condici\'on de salida.\end{frame}
\begin{frame}
\frametitle{Problema: Sin Nombre}
\noindent \textbf{Minimizar:}\\
\tab $Z = $ (2,00)\textcolor{yellow}{$x_{1}$} + (3,00)\textcolor{yellow}{$x_{2}$}\newline\newline
\textbf{Sujeto a:}\\
\tab (0,50)\textcolor{yellow}{$x_{1}$} + (0,25)\textcolor{yellow}{$x_{2}$} $\leq$ 4,00\newline\newline\tab \textcolor{yellow}{$x_{1}$} + (3,00)\textcolor{yellow}{$x_{2}$} $\geq$ 20,00\newline\newline\tab \textcolor{yellow}{$x_{1}$} + \textcolor{yellow}{$x_{2}$}= 10,00\newline\newline\end{frame}\begin{frame}
\frametitle{Tabla Inicial}
{
\centering
\begin{tabu}{|@{}*{8}{p{1.0cm}@{}|}}
\rowcolor{black}%
$Z$& $x_{1}$ & $x_{2}$ & $s_{1}$& $e_{1}$& $a_{1}$& $a_{2}$& result\\\hline
 1,00 & -2,00 & -3,00 & 0,00 & 0,00 & -M & -M & 0,00 \\\hline
 0,00 & 0,50 & 0,25 & 1,00 & 0,00 & 0,00 & 0,00 & 4,00 \\\hline
 0,00 & 1,00 & 3,00 & 0,00 & -1,00 & 1,00 & 0,00 & 20,00 \\\hline
 0,00 & 1,00 & 1,00 & 0,00 & 0,00 & 0,00 & 1,00 & 10,00 \\\hline
\rowcolor{black}%
\end{tabu}
}
\end{frame}\begin{frame}
\frametitle{Tabla Final}
{
\centering
\begin{tabu}{|@{}*{8}{p{1.0cm}@{}|}}
\rowcolor{black}%
$Z$& $x_{1}$ & $x_{2}$ & $s_{1}$& $e_{1}$& $a_{1}$& $a_{2}$& result\\\hline
 1,00 & 0,00 & 0,00 & 0,00 & -0,50 & 0,50 - M& 1,50 - M& 25,00 \\\hline
 0,00 & 0,00 & 0,00 & 1,00 & -0,12 & 0,12 & -0,62 & 0,25 \\\hline
 0,00 & 0,00 & 1,00 & 0,00 & -0,50 & 0,50 & -0,50 & 5,00 \\\hline
 0,00 & 1,00 & 0,00 & 0,00 & 0,50 & -0,50 & 1,50 & 5,00 \\\hline
\rowcolor{black}%
\end{tabu}
}
\end{frame}\begin{frame}
\frametitle{Solución}
\centering
{$Z = 25,00$\\
\textcolor{yellow}{$x_{1}$} $ = 5,00$\\
\textcolor{yellow}{$x_{2}$} $ = 5,00$\\
\textcolor{green}{$s_{1}$} $ = 0,25$\\
\textcolor{green}{$e_{1}$} $ = 0$\\
}
\end{frame}
\end{document}
